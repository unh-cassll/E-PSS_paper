\documentclass[14pt,letterpaper]{article}
\usepackage{nlaxague_course_document_style}
\renewcommand{\normalsize}{\fontsize{12}{14}\selectfont}

\usepackage{color}

\definecolor{violet}{RGB}{93,79,161}
\definecolor{teal}{RGB}{84,174,172}
\definecolor{sandy}{RGB}{253,190,110}
\definecolor{dark}{RGB}{100,100,100}
\definecolor{darkblue}{RGB}{0,0,128}

\definecolor{reddish}{RGB}{158,1,66}
\definecolor{yellowish}{RGB}{253,190,110}
\definecolor{greenish}{RGB}{117,199,164}

\lhead{} \chead{} \rhead{} \rfoot{Laxague \emph{et al.}, [2025]} \lfoot{Response to Reviewers}

\begin{document}

\noindent  \textbf{Response to Reviewers:\\ "The Effects of Subpixel Variability on Polarimetric Sensing of Ocean Waves"}\\

\noindent \textbf{Referee 1}\\

\textbf{Comments to the Author}\\
The manuscript analyses several features of the polarimetric sensing technique for the characterization of wave slopes of ocean waves in the application to the cameras with the novel Sony polarimetric sensor, where each pixel has four subpixels with different orientations of polarizers. The specific focus is on the spatial averaging of polarization state and possible aberrations in the slope field. The analysis is carried out in a very precise manner. For consistency, wave surfaces are generated by the computer code for conditions similar to those encountered in the field, polarization effects of the reflected light on the sensor are analyzed on these surfaces, and results are compared with field observations. The limitations of the models and analysis are clearly identified including the possible more complex structure of the waves in the field, the presence of currents and non-polarization from upwelling radiance. The latter assumption can be inaccurate in many life conditions but is reasonable for the modeling and analysis. Among others, an important result is that the differences between modeled and measured saturation spectra start from about $1/4$ of the Nyquist number (Fig. 11). Results are important for the further applications of the polarization sensing technique in above-water and airborne observations.
The manuscript can be published in its current state with at least one correction: Fig. 5 is missing.
\begin{itemize}
    \item \textcolor{darkblue}{We thank the reviewer for their time in considering our manuscript. We recognize that the placement of Figure 5 (top right of page 4) is confusing, as the figure appears above the full-width Figure 4 at the page bottom. This is a consequence of the order the figures appear in the text and are rendered in the \LaTeX~\hspace{-3pt} column environment, and we hope it will be resolved adequately should our manuscript be accepted and proceed to professional typesetting.}
\end{itemize}

\newpage

\noindent \textbf{Referee 2}\\

\textbf{Comments to the Author}\\
In this manuscript, two concerned areas associated with polarimetric slope sensing are identified. Such a study is significant for the tool PSS. It should clearly indicate some parameters and detail some of the processing.
\begin{itemize}
    \item \textcolor{darkblue}{We are grateful for the thoughtful comments provided by the reviewer. We respond to each enumerated concern below, with references given to specific changes made to the manuscript.}
\end{itemize}

% \vspace{20pt}

\begin{enumerate}
    \item Page 1, Right, lines 58-60 (C1): Two treatments (spatial averaging of polarization state and spatial averaging of surface orientation) are introduced directly, and they compare the results of these two treatments by simulation and field observations in Section IV-A, please elaborate on the steps and differences between the two treatments.
    \begin{itemize}
        \item \textcolor{darkblue}{We recognize that the steps/differences between the two types of averaging introduced in L58-60 of the manuscript bear heavily on the interpretation of results. The specific question following this text in L61-62 provides some context regarding the interpretation of C1. Furthermore, updates to section III-B should provide readers with a specific breakdown of the distinction between averaging in polarization state and averaging in slope.}
    \end{itemize}
    \item Page 3, lines 34-60 (Fig. 3): "degraded through averaging at the level of either (b) slope or (c) intensity", please specify the degradation process.
    \begin{itemize}
        \item \textcolor{darkblue}{We agree that we introduced confusion through the use of undefined terminology. The content of section III-B now spells this process out explicitly (with the help of Figure 6). The caption of Figure 3 now directs the reader to section III-B.}
    \end{itemize}
    \item Page 4, Right (Fig. 5): "The vectors \textbf{h} and \textbf{s} represent two byproducts which are used in the calculation of the polarization state of the reflected light". It's not clear how these two by-products are obtained and what they do. And explain how the Stokes vector of S relates to the normalized Stokes parameters.
    \begin{itemize}
        \item \textcolor{darkblue}{Regarding \textbf{h} and \textbf{s}, we agree that this level of detail is insufficient. The figure caption has been edited to properly describe \textbf{h} and \textbf{s}.}
        \item \textcolor{darkblue}{Regarding the normalized Stokes parameters, we have added text to the figure caption which connects the Stokes vector $\textbf{S}=[S_0,S_1,S_2]$ to the normalized Stokes parameters $\Tilde{S}_{1}=S_1/S_0$ and $\Tilde{S}_{2}=S_2/S_0$.}
    \end{itemize}
    \item Page 4 (Fig. 4): Is $\theta_i$ the angle of incident ray in Fig. 5?
    \begin{itemize}
        \item \textcolor{darkblue}{This is a good question/point to raise. Strictly speaking, $\theta$ is the incidence angle-- and we get one of those for each pixel (as the single facet shown in Figure 5 represents). But the panel title in Figure 4 makes reference to $\theta_i$, which in this context is the look angle of the modeled pinhole camera. So we have changed $\theta_i$ to $\theta_{look}$ in the panel title and defined this angle in the caption text.}
    \end{itemize}
    \item Page 6, lines 24-60 (Fig. 7): What are the "five quantiles" means? In this figure, what are the differences and similarities between the law of simulation results and the law of measured results? Please elaborate.
    \begin{itemize}
        \item \textcolor{darkblue}{With respect to the first point: in this context, the quantiles are the 10th, 25th, 50th, 75th, and 90th percentile of RMSE; organization of the data in this manner depicts the the median trend as well as the range of variation within a given block size.}
        \item \textcolor{darkblue}{With respect to the second point: we aim to emphasize two differences between the simulation results and the measured results in Figure 7. First, the lower sub-block RMS slope in observations than in simulation results is stark in Figure 7, and we provide two possible explanations for this difference (lines 350-370). Second, the peak in RMSE around a block size of 5-10mm is a more subtle feature of Figure 7, but reveals an important nuance in the simulated results, which is investigated further in Figure 8 and related discussion.}
    \end{itemize}
    \item Page 11, Right, lines 632-643: To make it easier for the author to read, it is recommended to advance this section to the end of page 11.
    \begin{itemize}
        \item \textcolor{darkblue}{We understand the reviewer's concern regarding the placement of text at the start of Section VI. However, we believe that the Conclusions of the manuscript are most effectively presented through the original flow, as listed below:
        \begin{enumerate}
            \item Review the two driving concerns which motivated the present work
            \item Recapitulate the activities performed in addressing those concerns
            \item Summarize the key findings in bulleted list form
            \item Provide a short takeaway message in paragraph form
        \end{enumerate}
        For this reason---and because the reviewer's comment does not pertain to the scientific methods or findings of the manuscript---we have not made any revision here.}
    \end{itemize}
    \item Page 2, Left, lines 12-16: The symbol for "I" is undefined.
    \begin{itemize}
        \item \textcolor{darkblue}{Here, "I" refers to light intensity. A parenthetical has been added for the purpose of explicitly stating this for the reader.}
    \end{itemize}
    \item Page 6, Right, line 323: What does N=70 mean?
    \begin{itemize}
        \item \textcolor{darkblue}{This is the number of cases considered following stringent quality control. "$N=70$" has been replaced with "70 cases" to remove the ambiguity.}
    \end{itemize}
    \item Page 7, Right, lines 348-355: The sub-block RMS slope obtained from ASIT field observations is substantially lower than the corresponding value in the simulation. One of the explanations they gave was that the disparity lies in the differing resolutions between the simulation ($\approx$0.3 mm px$^{-1}$) and field observations ($\approx$1.1 mm px$^{-1}$). Since it is a simulation, why can't the resolution be the same as the measured one, and then compare it?
    \begin{itemize}
        \item \textcolor{darkblue}{This is a good question. In short-- limitations of the surface reflection model prevent robust estimation of the Stokes parameters if the simulated surface is not sufficiently smooth. At a simulated pixel size of $\approx$1 mm, non-negligible energy in the wave spectrum at that scale yields a choppy surface which varies in slope to a great degree from one pixel to the next. This results result in large discontinuities in the modeled Stokes parameters. So, we kept the simulation resolution high (small virtual pixel size) in order to minimize the negative impact of those artifacts in the computations. We might have been able to mitigate the effect of those discontinuities through additional spatial smoothing of the low resolution slope fields before execution of the light reflection model. This would have improved the qualitative appearance of the fields at the expense of an extra level of smoothing that was not executed elsewhere. It may be that improvements to the light reflection model---e.g., making it a true "ray tracing" model which accounts for multiple surface reflections and volumetric scattering within the ocean surface layer---but we decided that a simpler model was the best fit for the present manuscript. Text describing our thought process on this matter has been added to page 7.}
    \end{itemize}
    \item Page 8, Right, lines 410-411, $k_{min}$ and $k_{max}$ are not defined. Figure 10 heavily depends on the definitions of these two wavenumbers.
    \begin{itemize}
        \item \textcolor{darkblue}{These correspond to the minimum and maximum wavenumber for each spectrum. $k_{min}$ is the same for all cases; $k_{max}$ corresponds to the Nyquist wavenumber for the full-resolution case, but decreases with the level of block-averaging. Text to this effect has been added to the right column of page 8.}
    \end{itemize}
    \item Page 9 (Fig. 11):  How is the saturation spectrum produced from the wave slope fields? Please elaborate. 
    \begin{itemize}
        \item \textcolor{darkblue}{The omnidirectional saturation spectrum $B(k)$ is produced from the omnidirectional slope spectrum $S(k)$ via $B(k)=kS(k)$. We have stated this explicitly within the text on page 8, also adding references to relevant literature.}
    \end{itemize}
\end{enumerate}

\end{document}